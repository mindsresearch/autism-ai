\documentclass[twoside, letterpaper, twocolumn]{article}

\title{Presenter Notes: On the Dangers of AI Autism Diagnosis}
\author{Noah Duggan Erickson\thanks{Department of Computer Science, College of Science ant Technology, Western Washington University}}
\date{7 March 2024}

\begin{document}
\maketitle
\section{Overview 1}
\begin{itemize}
    \item Getting started
    \item Technical, free to ask questions
\end{itemize}
\section{What is AIAD?}
\begin{itemize}
    \item Rigor involved in getting an ASD diagnosis
    \item A rapidly growing field in machine learning research
    \item EBSCO search for \texttt{AI (AND) Autism (AND) diagno*} gave 65 results: \textbf{20} of them are from \textbf{2023}, \textbf{10} from \textbf{2022}.
    \item Trying\dots \textit{trying\dots} to employ various machine learning techniques in the detection of ASD
\end{itemize}
\newpage
\section{Problems in General ML Research}
\begin{itemize}
    \item Taking a step back from AIAD for a second.
    \item What are the problem areas in general ML research?
    \item Models are growing faster than technology
    \item Increasing model size means increasing scale of data needed
    \item In the current hot topic of LLMs such as ChatGPT, where does the data come from? How is it filtered? What bias is intrinsic to the dataset? 
    \item No such thing as a perfect filter. Demographics of who edits Wikipedia?
    \item In a \textbf{2020 UW} study, GPT-2 produced toxic responses to \textit{at best} 17\% of innocuous prompts, even after the best ``conditioning'' the study performed.
\end{itemize}
\section{Disability Invisibility}
\begin{itemize}
    \item Won't spend too much time here\dots
    \item Treatment of disabled as sub-human, lesser
    \item In much of ML research, gender and race are frequently ``protected characteristics''. Disability often unmarked; \textit{maybe} a footnote.
    \item Conventional AI bias metrics actively blind to ableism; ``fairness'', tying into WW-II masculinity(?)
\end{itemize}
\section{Defining Autism}
\begin{itemize}
    \item How is autism defined in society?
    \item People who I shall not say their names aloud
    \item Indirect inhumanity
    \item AI ethics predicated on personhood; ``viable knowers''
\end{itemize}
\section{Current AIAD Literature}
\begin{itemize}
    \item What are researchers currently doing?
    \item Two main tracks: the MRI/EEG-based approach, the behavioral approach, and their intersection.
\end{itemize}
\subsection{MRI}
\begin{itemize}
    \item Most studies use the ABIDE dataset, an aggregation of 3D MRI brain scans from various institutions.
    \item The average from a 2021 review shows that the models ``get it correct'' about 78\% of the time.
\end{itemize}
\subsection{Behavioral}
\begin{itemize}
    \item Essentially trying to streamline traditional assessments
    \item Perpetuating the same biases as those assessments
\end{itemize}
\subsection{Intersection}
\begin{itemize}
    \item Computer vision on stimming behaviors
    \item eye tracking
    \item NLP
\end{itemize}
\section{Problems in AIAD}
\begin{itemize}
    \item Conclusions, wrap up
\end{itemize}
\end{document}
